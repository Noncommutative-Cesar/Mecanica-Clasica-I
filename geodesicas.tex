\documentclass[paper=a4, fontsize=11pt,twoside]{scrartcl}
\usepackage[a4paper,pdftex]{geometry}
\setlength{\oddsidemargin}{5mm}			
\setlength{\evensidemargin}{5mm}

\usepackage[spanish,activeacute]{babel}
\usepackage[protrusion=true,expansion=true]{microtype}	
\usepackage{amsmath,amsfonts,amsthm,amssymb}
\usepackage{graphicx}
\usepackage[section]{placeins}
\providecommand{\abs}[1]{\lvert#1\rvert}
\providecommand{\norm}[1]{\lVert#1\rVert}

\begin{document}

\section{Curvas Geod\'esicas.}

\subsection{Ejemplo en $\mathbb{R}^{2}$.}

(ARREGLAR INTRODUCCION)

Ya vimos que podemos usar el calculo variacional para obtener las ecuaciones del movimiento. Pero esta no es la \'unica aplicaci\'on que 
tiene el calculo de variaciones. Como vimos, consiste en perturbar continuamente una curva (usando un parametro) y encontrar la curva que 
extremize la acci'on. As\'i podemos usar el principio variacional para encontrar la curva de minima distancia entre dos puntos. Veamos 
primero el siguiente ejemplo para entender la situacion: \\

En el espacio Euclidiano $\mathbb{R}^{2}$ el elemento de linea esta dado por:
	\begin{equation*}
		ds^{2} = dx^{2} + dy^{2}
	\end{equation*}

Recordando que $s$ es la distancia entre dos puntos del espacio, digamos $P$ y $Q$, para $\mathbb{R}^{2}$, la distancia entre 
$P$ y $Q$ est\'a dada por:
	\begin{equation*}
		s = \int^{Q}_{P} ds = \int^{Q}_{P} \sqrt{dx^{2} + dy^{2}}
	\end{equation*}

Esta ser\'a nuestra acci\'on a extremar. \\

Ahora consideremos una curva cualquiera (con extremos los puntos $P$ y $Q$) en el espacio, parametrizada con un parametro 
$\lambda$, tal que:
	\begin{align*}
		x &= x \left( \lambda \right) \\
		y &= y \left( \lambda \right)
	\end{align*}

Luego, de la regla de la cadena tenemos que:
	\begin{align*}
		dx &= \frac{dx}{d\lambda} d\lambda \\
		dy &= \frac{dy}{d\lambda} d\lambda
	\end{align*}
	
Y definimos:
	\begin{align*}
		Q &= \left( x \left(\lambda_{f} \right) , y \left(\lambda_{f} \right) \right) \\
		P &= \left( x \left(\lambda_{i} \right) , y \left(\lambda_{i} \right) \right)
	\end{align*}
	
Usaremos la notaci\'on:
	\begin{align*}
		x' &= \frac{dx}{d\lambda} \\
		y' &= \frac{dy}{d\lambda}
	\end{align*}

Reemplazando esto en la acci\'on, tenemos:
	\begin{equation*}
		s = \int^{\lambda_{f}}_{\lambda_{i}} \sqrt{x'^{2} + y'^{2}}d\lambda
	\end{equation*}	
		
Ahora apliquemos la variaci\'on:
	\begin{align*}
		\delta s &= \delta \int^{\lambda_{f}}_{\lambda_{i}}  \sqrt{ x'^{2} + y'^{2} }d\lambda \\
				 &= \int^{\lambda_{f}}_{\lambda_{i}}  \delta \sqrt{ x'^{2} + y'^{2} }d\lambda \\
				 &= \int^{\lambda_{f}}_{\lambda_{i}}  \left( \frac{x' \delta x'}{\sqrt{ x'^{2} + y'^{2} }} + \frac{y' \delta y'}{\sqrt{ x'^{2} + y'^{2} }} \right) d\lambda \\
				 &= \int^{\lambda_{f}}_{\lambda_{i}}  \frac{x'}{\sqrt{ x'^{2} + y'^{2} }} \delta x' d\lambda 
				  + \int^{\lambda_{f}}_{\lambda_{i}}  \frac{y'}{\sqrt{ x'^{2} + y'^{2} }} \delta y' d\lambda
	\end{align*}

Trabajaremos con el termino que:
	\begin{equation*}
		\int^{\lambda_{f}}_{\lambda_{i}}  \frac{x'}{\sqrt{ x'^{2} + y'^{2} }} \delta x' d\lambda  
	\end{equation*}

Recordemos que:
	\begin{equation*}
		\delta x'= \delta \frac{dx}{d\lambda} = \frac{d}{d\lambda} \left( \delta x \right) = \frac{d \delta x}{d\lambda}
	\end{equation*}

As\'i:
	\begin{equation*}
		\int^{\lambda_{f}}_{\lambda_{i}}  \frac{x'}{\sqrt{ x'^{2} + y'^{2} }} \delta x' d\lambda =  \int^{\lambda_{f}}_{\lambda_{i}}  \frac{x'}{\sqrt{ x'^{2} + y'^{2} }} \frac{d \delta x}{d\lambda} d\lambda 
	\end{equation*}

Veamos que:
	\begin{equation*}
		\frac{x'}{\sqrt{ x'^{2} + y'^{2} }} \frac{d \delta x}{d\lambda} = \frac{d}{d\lambda} \left( \frac{x'}{\sqrt{ x'^{2} + y'^{2} }} \delta x  \right) 
																		  - \frac{d}{d\lambda} \left( \frac{x'}{\sqrt{ x'^{2} + y'^{2} }}  \right) \delta x 
	\end{equation*}

Luego:
	\begin{align*}
		\int^{\lambda_{f}}_{\lambda_{i}}  \frac{x'}{\sqrt{ x'^{2} + y'^{2} }} \frac{d \delta x}{d\lambda} d\lambda 
		  &= \int^{\lambda_{f}}_{\lambda_{i}}  \frac{d}{d\lambda} \left( \frac{x'}{\sqrt{ x'^{2} + y'^{2} }} \delta x  \right) d\lambda
	   	  - \int^{\lambda_{f}}_{\lambda_{i}}  \frac{d}{d\lambda} \left( \frac{x'}{\sqrt{ x'^{2} + y'^{2} }}  \right) \delta x d\lambda \\
	   	  &= \left( \frac{x'}{\sqrt{ x'^{2} + y'^{2} }} \delta x \right) \Big|^{\lambda_{f}}_{\lambda_{i}} - \int^{\lambda_{f}}_{\lambda_{i}}  \frac{d}{d\lambda} \left( \frac{x'}{\sqrt{ x'^{2} + y'^{2} }}  \right) \delta x d\lambda \\
	\end{align*}

Y como:
	\begin{equation*}
		\delta x (\lambda_{f}) = \delta x (\lambda_{i}) = 0 
	\end{equation*}

Obtenemos que:
	\begin{align*}
		\int^{\lambda_{f}}_{\lambda_{i}}  \frac{x'}{\sqrt{ x'^{2} + y'^{2} }} \frac{d \delta x}{d\lambda} d\lambda = - \int^{\lambda_{f}}_{\lambda_{i}}  \frac{d}{d\lambda} \left( \frac{x'}{\sqrt{ x'^{2} + y'^{2} }}  \right) \delta x d\lambda 
	\end{align*}

Por lo tanto:
	\begin{align*}
		\int^{\lambda_{f}}_{\lambda_{i}}  \frac{x'}{\sqrt{ x'^{2} + y'^{2} }} \delta x' d\lambda = - \int^{\lambda_{f}}_{\lambda_{i}}  \frac{d}{d\lambda} \left( \frac{x'}{\sqrt{ x'^{2} + y'^{2} }}  \right) \delta x d\lambda 
	\end{align*}
	
Analogamente:
	\begin{align*}
		\int^{\lambda_{f}}_{\lambda_{i}}  \frac{y'}{\sqrt{ x'^{2} + y'^{2} }} \delta y' d\lambda = - \int^{\lambda_{f}}_{\lambda_{i}}  \frac{d}{d\lambda} \left( \frac{y'}{\sqrt{ x'^{2} + y'^{2} }}  \right) \delta y d\lambda 
	\end{align*}

Por lo que:
	\begin{equation*}
		\delta s = - \int^{\lambda_{f}}_{\lambda_{i}}  \frac{d}{d\lambda} \left( \frac{x'}{\sqrt{ x'^{2} + y'^{2} }}  \right) \delta x d\lambda  - \int^{\lambda_{f}}_{\lambda_{i}}  \frac{d}{d\lambda} \left( \frac{y'}{\sqrt{ x'^{2} + y'^{2} }}  \right) \delta y d\lambda 
	\end{equation*}	

Si pedimos que $\delta s = 0$ (buscamos los extremos), tenemos que:
	\begin{align*}
		- \int^{\lambda_{f}}_{\lambda_{i}}  \frac{d}{d\lambda} \left( \frac{x'}{\sqrt{ x'^{2} + y'^{2} }}  \right) \delta x d\lambda  - \int^{\lambda_{f}}_{\lambda_{i}}  \frac{d}{d\lambda} \left( \frac{y'}{\sqrt{ x'^{2} + y'^{2} }}  \right) \delta y d\lambda = 0
	\end{align*}

Y como $\delta x$ y  $\delta y$ son linealmente independientes, tenemos que:
	\begin{align*}
		\frac{d}{d\lambda} \left( \frac{x'}{\sqrt{ x'^{2} + y'^{2} }}  \right) &= 0 \\
		\frac{d}{d\lambda} \left( \frac{y'}{\sqrt{ x'^{2} + y'^{2} }}  \right) &= 0
	\end{align*}

De donde:
	\begin{align*}
		 \frac{x'}{\sqrt{ x'^{2} + y'^{2} }}  &= A \\
		 \frac{y'}{\sqrt{ x'^{2} + y'^{2} }}  &= B
	\end{align*}

Con $A$ y $B$ constantes. Dividimos las expresiones y obtenemos:
	\begin{align*}
		\frac{y'}{x'} = C
	\end{align*}

Y como:
	\begin{equation*}
		\frac{y'}{x'} = \frac{dy/d\lambda}{dx/d\lambda} = \frac{dy}{d\lambda} \frac{d\lambda}{dx} = \frac{dy}{dx}
	\end{equation*}

Finalmente:
	\begin{align*}
		\frac{dy}{dx} &= C \\
		\Rightarrow y &= Cx + D
	\end{align*}
	
Lo que corresponde a una recta. \\

En resumen, perturbamos una curva en el espacio Euclidiano $\mathbb{R}^{2}$ y aplicamos el principio variacional para encontrar la curva de 
longitud minima entre dos puntos, obteniendo as\'i una recta, lo cual era el resultado esperado. \\

\textbf{Nota:} El calculo fue realizado de la manera m\'as general posible, no necesariamente siendo esto la manera m\'as \'optima de 
aplicar la variaci\'on. Sugerimos rehacer el calculo, pero esta vez, usando el teorema de la funci\'on implicita, para obtener la ecuaci\'on 
de la curva de la forma $y=y(x)$, y luego hacer variar la acci\'on. Se dar\'a cuenta de que realizar el calculo de esta forma es mucho m\'as 
sencillo (pero no tan general). \\

\textit{Hint:} Debe obtener el mismo resultado anterior :)

\subsection{Curvas Geod\'esicas en cualquier espacio.}

Una vez entendido la manera de abordar este problema, nos gustar\'ia saber cual es la curva de minima longitud que une dos puntos, 
esta vez no necesariamente en $\mathbb{R}^{2}$, sino que, en \textbf{cualquier espacio}, de \textbf{cualquier dimensi\'on}. \\

Empecemos considerando un espacio de $n$ dimensiones, con una m\'etrica $g_{ij}$, donde $i,j = 1,...,n$. El elemento de linea en este espacio 
esta dado por:
	\begin{equation*}
		ds^{2} = g_{ij}dx^{i}dx^{j}
	\end{equation*}

Como en el ejemplo, consideremos dos puntos ($P$ y $Q$) en el espacio, la distancia entre ellos esta dada por:
	\begin{equation*}
		s = \int^{Q}_{P} ds = \int^{Q}_{P} \sqrt{g_{ij}dx^{i}dx^{j}}
	\end{equation*}

Siguiendo con la metodolog\'ia usada en el ejemplo, consideremos una curva en el espacio, que tiene por extremos los puntos $P$ y $Q$, 
y que ademas esta parametrizada con un parametro $\lambda$, tal que:
	\begin{equation*}
		x^{i} = x^{i} (\lambda)
	\end{equation*}

De donde:
	\begin{equation*}
		dx^{i} = \frac{dx^{i}}{d\lambda} d\lambda = x'^{i}d\lambda
	\end{equation*}
	
Y definimos:
	\begin{align*}
		Q &= \left( x^{1} \left(\lambda_{f} \right) ,...,  x^{n} \left(\lambda_{f} \right) \right) \\
		P &= \left( x^{1} \left(\lambda_{i} \right) ,...,  x^{n} \left(\lambda_{i} \right) \right)
	\end{align*}
	
Notemos que:
	\begin{equation*}
		\frac{ds}{d\lambda} = \sqrt{g_{ij}x'^{i}x'^{j}}
	\end{equation*}

Reemplazando en la acci\'on:
	\begin{equation*}
		s = \int^{\lambda_{f}}_{\lambda{i}} \sqrt{g_{ij}x'^{i}x'^{j}} d\lambda
	\end{equation*}

Aplicando la varici\'on:
	\begin{align*}
		\delta s &= \delta\int^{\lambda_{f}}_{\lambda{i}} \sqrt{g_{ij}x'^{i}x'^{j}} d\lambda \\
				 &= \int^{\lambda_{f}}_{\lambda{i}} \delta \sqrt{g_{ij}x'^{i}x'^{j}} d\lambda \\
				 &= \int^{\lambda_{f}}_{\lambda{i}} \frac{\delta\left( g_{ij}x'^{i}x'^{j} \right)}{2\sqrt{g_{ab}x'^{a}x'^{b}}} d\lambda \\
				 &= \int^{\lambda_{f}}_{\lambda{i}} \frac{\delta g_{ij}x'^{i}x'^{j} + 2g_{ij}x'^{i}\delta x'^{j}}{2\sqrt{g_{ab}x'^{a}x'^{b}}} d\lambda \\
	\end{align*}
	
Notemos que:
	\begin{equation*}
		\delta g_{ij} = \frac{\partial g_{ij}}{\partial x^{k}} \delta x^{k}
	\end{equation*}
	
As\'i: 
	\begin{align*}
		\delta s &= \int^{\lambda_{f}}_{\lambda{i}} \frac{1}{2\sqrt{g_{ab}x'^{a}x'^{b}}} \left(  \frac{\partial g_{ij}}{\partial x^{k}}x'^{i}x'^{j} \delta x^{k} + 2g_{ij}x'^{i}\delta x'^{j} \right) d\lambda \\
				 &= \int^{\lambda_{f}}_{\lambda{i}} \left( \frac{\frac{\partial g_{ij}}{\partial x^{k}}x'^{i}x'^{j}}{2\sqrt{g_{ab}x'^{a}x'^{b}}} \delta x^{k}
				                                        +  \frac{g_{ij}x'^{i}\delta x'^{j}}{\sqrt{g_{ab}x'^{a}x'^{b}}} \right) d\lambda 
	\end{align*}

Veamos el termino:
	\begin{equation*}
		\frac{g_{ij}x'^{i}\delta x'^{j}}{\sqrt{g_{ab}x'^{a}x'^{b}}}
	\end{equation*}

Recordemos que:
	\begin{equation*}
		\delta x'^{i} = \frac{d \delta x^{i}}{d\lambda}
	\end{equation*}

As\'i:
	\begin{align*}
		\frac{g_{ij}x'^{i}\delta x'^{j}}{\sqrt{g_{ab}x'^{a}x'^{b}}} &= \frac{g_{ij}x'^{i}}{\sqrt{g_{ab}x'^{a}x'^{b}}}\frac{d \delta x^{j}}{d\lambda} \\
																	&= \frac{d}{d\lambda} \left(\frac{g_{ij}x'^{i}}{\sqrt{g_{ab}x'^{a}x'^{b}}} \delta x^{j}\right) 
																	 - \frac{d}{d\lambda} \left( \frac{g_{ij}x'^{i}}{\sqrt{g_{ab}x'^{a}x'^{b}}} \right)\delta x^{j}
	\end{align*}
	
Luego:
	\begin{align*}
		\int^{\lambda_{f}}_{\lambda{i}} \frac{g_{ij}x'^{i}\delta x'^{j}}{\sqrt{g_{ab}x'^{a}x'^{b}}} d\lambda
			&= \int^{\lambda_{f}}_{\lambda{i}} \frac{d}{d\lambda} \left(\frac{g_{ij}x'^{i}}{\sqrt{g_{ab}x'^{a}x'^{b}}} \delta x^{j}\right) d\lambda
			 - \int^{\lambda_{f}}_{\lambda{i}} \frac{d}{d\lambda} \left( \frac{g_{ij}x'^{i}}{\sqrt{g_{ab}x'^{a}x'^{b}}} \right)\delta x^{j} d\lambda \\
			&= \left( \frac{g_{ij}x'^{i}}{\sqrt{g_{ab}x'^{a}x'^{b}}} \delta x^{j} \right) \Big|^{\lambda_{f}}_{\lambda_{i}} 
			- \int^{\lambda_{f}}_{\lambda{i}} \frac{d}{d\lambda} \left( \frac{g_{ij}x'^{i}}{\sqrt{g_{ab}x'^{a}x'^{b}}} \right) \delta x^{j} d\lambda
	\end{align*}
	
Y como:
	\begin{equation*}
		\delta x^{i} (\lambda_{i}) = \delta x^{i} = 0
	\end{equation*}

Tenemos que:
	\begin{equation*}
		\int^{\lambda_{f}}_{\lambda{i}} \frac{g_{ij}x'^{i}\delta x'^{j}}{\sqrt{g_{ab}x'^{a}x'^{b}}} d\lambda
			= - \int^{\lambda_{f}}_{\lambda{i}} \frac{d}{d\lambda} \left( \frac{g_{ij}x'^{i}}{\sqrt{g_{ab}x'^{a}x'^{b}}} \right) \delta x^{j} d\lambda
	\end{equation*}
		
Luego:
	\begin{align*}
		\delta s &= \int^{\lambda_{f}}_{\lambda{i}} \left( \frac{\frac{\partial g_{ij}}{\partial x^{k}}x'^{i}x'^{j}}{2\sqrt{g_{ab}x'^{a}x'^{b}}} \delta x^{k}
				                                        -  \frac{d}{d\lambda} \left( \frac{g_{ij}x'^{i}}{\sqrt{g_{ab}x'^{a}x'^{b}}} \right) \delta x^{j} \right) d\lambda \\ 
				 &= \int^{\lambda_{f}}_{\lambda{i}} \frac{1}{\sqrt{g_{ab}x'^{a}x'^{b}}} \left( \frac{1}{2}\frac{\partial g_{ij}}{\partial x^{k}} x'^{i}x'^{j} \delta x^{k}
				                                        -  \sqrt{g_{cd}x'^{c}x'^{d}}\frac{d}{d\lambda} \left( \frac{g_{ij}x'^{i}}{\sqrt{g_{ef}x'^{e}x'^{f}}} \right) \delta x^{j} \right) d\lambda \\ 
				 &= \int^{\lambda_{f}}_{\lambda{i}} \frac{1}{\sqrt{g_{ab}x'^{a}x'^{b}}} \left( \frac{1}{2}\frac{\partial g_{ij}}{\partial x^{k}} x'^{i}x'^{j} \delta x^{k}
				                                        - \frac{d}{d\lambda} \left( g_{ij}x'^{i} \right) \delta x^{j}  
				                                        + \frac{g_{ij}x'^{i}}{\sqrt{g_{ef}x'^{e}x'^{f}}} \frac{d}{d\lambda} \left( \sqrt{g_{cd}x'^{c}x'^{d}} \right) \delta x^{j} 
														\right) d\lambda\\
				&= 	\int^{\lambda_{f}}_{\lambda{i}} \frac{1}{\sqrt{g_{ab}x'^{a}x'^{b}}} \left( \frac{1}{2}\frac{\partial g_{ij}}{\partial x^{k}} x'^{i}x'^{j} \delta x^{k}
				                                        - \frac{\partial g_{ij}}{\partial x^{k}} x'^{i} x'^{k}\delta x^{j} - g_{ij}x''^{i}\delta x^{j}  
				                                        + \frac{g_{ij}x'^{i}}{\sqrt{g_{ef}x'^{e}x'^{f}}} \frac{d}{d\lambda} \left( \sqrt{g_{cd}x'^{c}x'^{d}} \right) \delta x^{j} 
														\right) d\lambda										
	\end{align*}
	
Notemos que:
	\begin{equation*}
		\frac{1}{\sqrt{g_{ef}x'^{e}x'^{f}}} \frac{d}{d\lambda} \left( \sqrt{g_{cd}x'^{c}x'^{d}} \right)
		=  \frac{d}{d\lambda} \left( \ln \left( \sqrt{g_{cd}x'^{c}x'^{d}} \right) \right)
	\end{equation*}

Definiendo:
	\begin{equation*}
		\alpha(s) = \frac{d}{d\lambda} \left( \ln \left( \sqrt{g_{cd}x'^{c}x'^{d}} \right) \right)
	\end{equation*}
	
Reemplazando en $\delta s$ y cambiando el indice de suma:
	\begin{equation*}
		\delta s =  \int^{\lambda_{f}}_{\lambda{i}} \frac{1}{\sqrt{g_{ab}x'^{a}x'^{b}}} \left( \frac{1}{2}\frac{\partial g_{ij}}{\partial x^{k}} x'^{i}x'^{j} 
			                                       - \frac{\partial g_{ik}}{\partial x^{j}} x'^{i} x'^{j} - g_{ik}x''^{i} 
				                                        + \alpha\left( s \right) g_{ik}x'^{i} 
														\right) \delta x^{k}  d\lambda	
	\end{equation*}
	
\newpage
	
Notemos que (ejercicio):
	\begin{equation*}
		\frac{\partial g_{ik}}{\partial x^{j}} x'^{i} x'^{j} = \frac{1}{2} \left( \frac{\partial g_{ik}}{\partial x^{j}} + \frac{\partial g_{jk}}{\partial x^{i}} \right) x'^{i}x'^{j}
	\end{equation*}

\textit{Hint:} Tensores sim\'etricos y antisim\'etricos.\\

Reemplazando lo anterior:
	\begin{align*}
		\delta s &=  \int^{\lambda_{f}}_{\lambda{i}} \frac{1}{\sqrt{g_{ab}x'^{a}x'^{b}}} \left( \frac{1}{2}\frac{\partial g_{ij}}{\partial x^{k}} x'^{i}x'^{j} 
				                                    - \frac{1}{2} \left( \frac{\partial g_{ik}}{\partial x^{j}} + \frac{\partial g_{jk}}{\partial x^{i}} \right) x'^{i}x'^{j} - g_{ik}x''^{i} 
				                                        + \alpha\left( s \right) g_{ik}x'^{i} 
														\right) \delta x^{k}  d\lambda\\	
				 &=  \int^{\lambda_{f}}_{\lambda{i}} \frac{1}{\sqrt{g_{ab}x'^{a}x'^{b}}} \left( 
													\frac{1}{2}\frac{\partial g_{ij}}{\partial x^{k}} x'^{i}x'^{j} 
				                                    - \frac{1}{2}\frac{\partial g_{ik}}{\partial x^{j}}x'^{i}x'^{j} - \frac{1}{2}\frac{\partial g_{jk}}{\partial x^{i}} x'^{i}x'^{j} - g_{ik}x''^{i} 
				                                        + \alpha\left( s \right) g_{ik}x'^{i} 
														\right) \delta x^{k}  d\lambda \\
				 &=  \int^{\lambda_{f}}_{\lambda{i}} \frac{1}{\sqrt{g_{ab}x'^{a}x'^{b}}} \left( - g_{ik}x''^{i}
													-\frac{1}{2}\left( \frac{\partial g_{ik}}{\partial x^{j}} + \frac{\partial g_{jk}}{\partial x^{i}}-\frac{\partial g_{ij}}{\partial x^{k}} \right)x'^{i}x'^{j}
				                                        + \alpha \left( s \right) g_{ik}x'^{i} 
														\right) \delta x^{k}  d\lambda \\
				 &=  \int^{\lambda_{f}}_{\lambda{i}} \frac{1}{\sqrt{g_{ab}x'^{a}x'^{b}}} \left( - g_{il}x''^{i}
													-\frac{1}{2}\left( \frac{\partial g_{kl}}{\partial x^{j}} + \frac{\partial g_{jl}}{\partial x^{k}}-\frac{\partial g_{kj}}{\partial x^{l}} \right)x'^{j}x'^{k}
				                                        + \alpha\left( s \right) g_{il}x'^{i} 
														\right) \delta x^{l}  d\lambda \\	
				 &=  \int^{\lambda_{f}}_{\lambda{i}} \frac{1}{\sqrt{g_{ab}x'^{a}x'^{b}}} \left( - g_{il}x''^{i}
													-\frac{1}{2}\delta^{m}_{l}\left( \frac{\partial g_{km}}{\partial x^{j}} + \frac{\partial g_{jm}}{\partial x^{k}}-\frac{\partial g_{kj}}{\partial x^{m}} \right)x'^{j}x'^{k}
				                                        + \alpha\left( s \right) g_{il}x'^{i} 
														\right) \delta x^{l}  d\lambda \\																																												
	\end{align*}

Pero:
	\begin{equation*}
		\delta^{m}_{l} = g_{il}g^{im}
	\end{equation*}

Tenemos que:
	\begin{align*}
		\delta s &=  \int^{\lambda_{f}}_{\lambda{i}} \frac{1}{\sqrt{g_{ab}x'^{a}x'^{b}}} \left( - g_{il}x''^{i}
													-\frac{1}{2}g_{il}g^{im}\left( \frac{\partial g_{km}}{\partial x^{j}} + \frac{\partial g_{jm}}{\partial x^{k}}-\frac{\partial g_{kj}}{\partial x^{m}} \right)x'^{j}x'^{k}
				                                        + \alpha\left( s \right) g_{il}x'^{i} 
														\right) \delta x^{l}  d\lambda \\
			     &=  \int^{\lambda_{f}}_{\lambda{i}} \frac{g_{il}}{\sqrt{g_{ab}x'^{a}x'^{b}}} \left( - x''^{i}
													-\frac{1}{2}g^{im}\left( \frac{\partial g_{km}}{\partial x^{j}} + \frac{\partial g_{jm}}{\partial x^{k}}-\frac{\partial g_{kj}}{\partial x^{m}} \right)x'^{j}x'^{k}
				                                        + \alpha\left( s \right) x'^{i} 
														\right) \delta x^{l}  d\lambda \\																																												
	\end{align*}
	
Recordando que el simbolo de Christofell se define como:
	\begin{equation*}
		\Gamma^{i}_{jk} = \frac{1}{2} g^{im} \left( \frac{\partial g_{km}}{\partial x^{j}} + \frac{\partial g_{jm}}{\partial x^{k}}-\frac{\partial g_{kj}}{\partial x^{m}}    \right)
	\end{equation*}
	
Finalmente obtenemos que:
	\begin{equation*}
		\delta s =  \int^{\lambda_{f}}_{\lambda{i}} \frac{g_{il}}{\sqrt{g_{ab}x'^{a}x'^{b}}} \left( - x''^{i}
													-\Gamma^{i}_{jk} x'^{j}x'^{k} + \alpha\left( s \right) x'^{i} 
														\right) \delta x^{l}  d\lambda 																																											
	\end{equation*}
	
Pidiendo que $\delta s = 0$ y considerando que los $\delta x^{i}$ son linealmente independiente, obtenemos que:
	\begin{equation*}
		x''^{i} + \Gamma^{i}_{jk} x'^{j}x'^{k} = \alpha\left( s \right) x'^{i}
	\end{equation*}

Esta es llamada \textbf{Ecuaci\'on de la Geod\'esica}, esta es la curva de minima longitud entre dos puntos, que esta parametrizada 
con un parametro cualquiera $\lambda$. Pero esta ecuaci\'on puede ser bastante dif\'icil de resolver debido a la funci\'on $\alpha(s)$. 
Ser\'ia bastante comodo que esta funci\'on fuera igual a cero. Veamos que pasa si pedimos esa condici\'on:
	\begin{align*}
		\alpha(s) &= 0 \\
		\Rightarrow \frac{d}{d\lambda} \left( \ln \left( g_{ab} x'^{a}x'^{b}\right) \right) &= 0 \\
		\Rightarrow \frac{d}{d\lambda} \left( \ln \left( \frac{ds}{d\lambda }\right) \right) &= 0 \\
		\Rightarrow \frac{d\lambda}{ds}\frac{d^{2}s}{d\lambda^{2}} &= 0 \\
		\Rightarrow \frac{d^{2}s}{d\lambda^{2}} &= 0 \\
		\Rightarrow s &= a \lambda + b \\  
		\Rightarrow \lambda &= \frac{s}{a} - b \\
	\end{align*}

La longitud de arco $s$ puede ser usada como par\'ametro a lo largo de la curva. A $s$ se le llama par\'ametro natural, y a $x^{i} = x^{i}(s)$ 
se le llama parametrizaci\'on natural. Para el par\'ametro natural usaremos la siguiente notaci\'on:
	\begin{equation*}
		\dot{x}^{i} = \frac{dx^{i}}{ds}
	\end{equation*}

Si una curva esta parametrizada con el par\'ametro natural se cumple que:
	\begin{equation*}
		 \sqrt{g_{ij}\dot{x}^{i} \dot{x}^{j}} = 1
	\end{equation*}

Notemos que:
	\begin{align*}
		\frac{d}{d\lambda} &= \frac{d}{ds}\frac{ds}{d\lambda} \\ 
						   &= a\frac{d}{ds} \\
		\Rightarrow x'^{i} &= a \dot{x}^{i} \\
		\Rightarrow x''^{i} &= a^{2} \ddot{x}^{i} \\
 	\end{align*}
	
Reemplazando en la ecuaci\'on de la geod\'esica	obtenemos:
	\begin{equation*}
		\ddot{x}^{i} + \Gamma^{i}_{jk} \dot{x}^{j}\dot{x}^{k} = 0
	\end{equation*}

Que corresponde a la \textbf{ Ecuaci\'on de la Geod\'esica Af\'in}.

\end{document}


